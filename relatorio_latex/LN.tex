\documentclass{article}
\usepackage[utf8]{inputenc}
\usepackage[top=0.80in, bottom=1.00in, left=1.5in, right=1.5in]{geometry}
\newcommand\tab[1][1cm]{\hspace*{#1}}


\begin{document}
\title{Língua Natural - 1º Mini-projeto}
\author{Grupo 40 \and Sofia Aparício 81105 \and Rodrigo Lousada 81115}
\maketitle

\subsection*{Explicação do funcionamento}
 Existem 3 ficheiros que pode correr com o comando \textbf{sh}:
 \begin{itemize}
\item\textbf{run.sh} - Cria as pastas transducers, tests e transducers onde colocará todos os ficheiros gerados. No final copia os ficheiros referidos no enunciado para a pasta raiz com os nomes pedidos.
 \item\textbf{tests.sh} - Gera mais testes que estarão na pasta tests com os respectivos resultados na pasta results
 \item\textbf{clean.sh} - Limpa todos os ficheiros gerados
\end{itemize}

\subsection*{Descrição das Opções Tomadas e Comentários à solução}
\tab O grupo optou por começar a fazer o transdutor de Romanos para Decimais, sendo mais simples a implementação deste que depois seria invertido para obter o Transdutor Romanos. Simplificando mais ainda, começamos por desenvolver 4 transdutores, Referente ao zero, às unidades, às dezenas e ao numero 100. Estes quatro foram então concatenados e unidos de forma a obter um transdutor que contemplasse todos e somente os números de 1 a 100. \\
A necessidade de ter um transdutor apenas para o zero, deve-se ao facto do trandutor das unidades apenas poder contemplar o 0 caso já tenha sido lida uma dezena. Esta solução facilitaria-nos escalar assim a solução para o caso de aceitarmos numero de 0 a 999, evitar outputs como 00.\\

Para obter o trandutor 1 acabamos por criar mais dois transdutores: um trandutor que aceita todas as letras de a-z e um transdutor que aceita após receber o caracter \_(underline)\\
É obtido assim o transdutor 1 com a união entre o Transdutor Romanos, e o transdutor das letras com a posterior concatenação com o underline, e closure.\\

O transdutor 2 é um simples transdutor que apenas vai para o estado de aceitação caso receba um underline, e que ao receber qualquer numero Romano o codifica.\\

O transdutor 3 revela-se o mais desafiante, aplicando uma solução não determinista nos casos em que recebe um m ou um i. Decidimos construir o transdutor 3 de uma vez para uma solução mais eficiente, em vez de fazer a composição de vários transdutores mais pequenos que originariam uma solução final muito mais complexa.\\

O transdutor codificador é um simples compose entre os transdutor 1, 2 e 3 respectivamente, enquanto o transdutor descodificador resulta da inversão dos transdutores e respectivo compose pela ordem 3, 2 e 1.\\

O grupo optou, tal como aconselhado, a criar uma solução que parte de transdutores pequenos e simples para serem usados em transdutores mais complexos através da utilização das FSTools leccionadas, excepto quando julgou que não se justificaria.\\


\end{document}